%RELATED WORK
In this section, we mainly overview the existing work related to our KaGWC queries, focusing mostly on conventional spatial keyword queries and collective spatial keyword queries.

\subsection{Conventional Spatial Keyword Queries}
The conventional spatial keyword queries \cite{cong2009efficient, de2008keyword} take a location and a set of keywords as arguments, and return objects that can satisfy the users' needs solely. There are lots of efforts on conventional spatial keyword queries. By combining with existing queries in database community, there are several variants of conventional spatial keyword queries. We will review these queries as follows.

\textit{Combining with top-k queries}. By combining with the top-k queries, the top-k spatial keyword queries \cite{cong2009efficient, gaoefficient, li2011ir, rocha2012top, shang2012user, wu2012joint, wu2011efficient} retrieve k objects with the highest ranking scores by utilizing the ranking function, which takes both location and the relevance of textual descriptions into consideration. To address the top-k spatial keyword queries efficiently, various hybrid indexes have been explored. This branch includes \cite{cong2009efficient, li2011ir} (IR-tree), \cite{cary2010efficient} (SKI), \cite{rocha2011efficient, rocha2012top} (S2I). \cite{cong2012efficient, shang2012user} study the top-k spatial keyword queries over trajectory data. Cong et al. \cite{cong2012efficient} utilizes the hybrid index $B^{ck}$-tree to facilitate the query processing of top-k trajectories. Wu et al. \cite{wu2012joint} handles the joint top-k spatial keyword queries utilizing the W-IR-Tree index. Zhang et al. \cite{zhang2013scalable} demonstrates that $I^3$ index, which adopts the Quadtree structure to hierarchically partition the data space into cells, is superior to IR-tree and S2I. Gao et al. \cite{gaoefficient} studies the reverse top-k boolean spatial keyword queries on the road network with count tree.

\textit{Combining with NN queries}. The spatial keyword NN queries retrieve object that closes to the query location and contains the query keywords. Several variants have been explored. Tao et al. \cite{tao2014fast} proposes the new index called the SI-index to cope with multidimensional data, which overcomes the drawback of IR2-Tree \cite{de2008keyword} with Z-cuves. Lu et al. \cite{lu2011reverse} studies the RSTkNN query, finding the objects that take a specified query object as one of their k most spatial-textual similar objects.

\textit{Combining with route queries}. The conventional route queries \cite{li2005trip} in spatial database search the shortest route that starts at location s, passes through as least one object from each category in C and ends at t. Yao et al. \cite{yao2011multi} proposes the multi-approximate-keyword routing(MARK) query, which searches for the route with the shortest length such that it covers at least one matching object per keyword with the similarity larger than the corresponding threshold value. The problem of keyword-aware optimal route search (KOR) is studied in \cite{cao2012keyword}, to find the route which can cover a set of user-specified keywords, a specified budget constraint is satisfied, and an objective score of the route is optimal. Three algorithms are proposed for this problem in \cite{cao2012keyword}, and the corresponding system of KOR is provided in their subsequent work \cite{cao2013kors}.

\subsection{Collective Keyword Queries}
All these works aforementioned return object that can meet the users' needs solely. However, in real life applications, it is common to satisfy the users' needs collectively by a group of objects. The mCK queries \cite{zhang2009keyword, zhang2010locating} return a set of objects to cover the query keywords. However, in the context of mCK, each object associates with only a single keyword and it only takes keywords into consideration. The most similar work to ours is CoSKQ querry \cite{cao2011collective, long2013collective}. With the maximum sum cost function, Cao et al. \cite{cao2011collective} provides approximation algorithms as well as an exact algorithm. To further improve the performance, Long et al. \cite{long2013collective} proposes a distance owner-driven approach, besides they also propose a new cost measurement called diameter cost and design an exact and an approximation algorithms for it.

Although both CoSKQ and our queries retrieve a group of objects as result and take the keywords and location into consideration, however, our work differs with CoSKQ mainly in three aspects: 1) we take the level information of keyword into consideration, which is  crucial for users to make decisions; 2) we introduce weight vector to capture the users' preferences, which offers greater flexibility to its users when looking for interesting objects. 3) we take the combination of space distance and cost of object as our cost function, which is more closer to real life application scenarios. Due to these differences between CoSKQ and our KaGWC query, the methods adopted by CoSKQ can not be extended to solve our problem. Deng et al. \cite{deng2015best} studies the BKC query which considers the keyword rating information and returns best keyword cover, which is different with ours in the query goal. Besides, BKC does not utilize the weight vector to capture the users' preferences.

%problem statement
In this section, we first introduce the basic concepts used in this paper. Then we will prove the NP-hard complexity of our problem.

Let O denotes a set containing n spatial objects. Each object $o\in O$ is associated with a location $o.\ell$, a set of keywords $o.\omega$ to capture the tourist attractions or hotels etc. and an $|q.\omega|$ dimensions auxiliary vector $o.\nu$ to record the auxiliary information of the keywords (e.g., the level of tourist attractions or hotels), and each dimension corresponds to one keyword in $q.\omega$. For an object $o\in O$, we define the \textit{weight} of $o$ as the sum of each dimension value of auxiliary vector, i.e., $w(o)=\sum \limits_{i=1}^{|o.\nu|}o.\nu_i$ , which can represent the cost of $o$. For ease of presentation, in this work, we take the auxiliary vector as a multidimensional positive integer vector. And the upper bound of integer value is fixed (e.g., for attractions not larger than 5, that is to say only has 5 levels for attractions).

\noindent
\begin{definition}
    (Weight Distance) Given a query q and an object $o\in O$, the weight distance of $o$ can be denoted as:
    \begin{equation}
        wd(o,q)=w(o) \cdot dist(o,q)
    \end{equation}
\end{definition}

In this work, we take a combination of $w(o)$ and $dist(o,q)$, the distance between o and query point q, as cost function, which is closer to the real application scenarios than cost function used in [11, 26].

\noindent
\begin{definition}
    (Object coverage probability) Given a keyword $\lambda$, a multidimensional user preference vector P and an object $o\in O$. We use the notation $o.\nu_\lambda$ to denotes the corresponding integer value of in $o.\nu$. Then the probability that $\lambda$ covered by $o$ can be represented as:
    \begin{equation}
        cp(o,\lambda)=P[o.\nu_\lambda]
    \end{equation}
\end{definition}

Differing with CoSKQ in which a keyword either covered by object o or not, in this work we take the coverage probability \textit{cp} as our measurement. Note that if $\lambda$ is not contained by $o.\omega$, we set $cp(o,\lambda)=0$. We use $cov(o,q)=\sum \limits_{\lambda \in q.\omega}cp(o,\lambda)$ and $cov(G,\lambda)=\sum \limits_{o \in G}cp(o,\lambda)$ to denote the probability that q is covered by o and the probability that $\lambda$ is covered by G, respectively.

\noindent
\begin{definition}
    (Contribution ratio) Given a query q and an object o, we define the contribution ratio of the object o to q as follows:
    \begin{equation}
        cr(o,q)=\frac{cov(o,q)}{wd(o,q)}
    \end{equation}
\end{definition}
By taking both of the coverage probability and the weight distance cost into consideration, $cr(o,q)$ can fit in with the real application scenarios better than only use $cov(o,q)$ to evaluate the contribution of o to q.

\noindent
\begin{definition}
    (KaGPC queries) The KaGPC queries $q=(\ell,\theta,P,\omega)$, where $\ell$ is a spatial location and $\theta$ is a threshold. P is a probability model to capture user��s preference vector and $\omega$ represents the query keywords. KaGPC queries aim at retrieving a group G of objects that collectively satisfy the following two conditions:
    \begin{itemize}
        \item For each keyword $\lambda \in q.\omega$, $cov(G,\lambda)\geq q.\theta$;
        \item $\underaccent{G}{\arg} \min\sum \limits_{o \in G}wd(o,q)$.
    \end{itemize}
\end{definition}

Given a KaGPC queries q, we say an object o is \textit{\textbf{relevant}} to q if o contains at least one keyword $\lambda \in q.\omega$. We use notation \textbf{$RO_\lambda$} and \textbf{$RO_q$} to denote the set of objects that relevant to $\lambda$ and query q in O, respectively. It��s sufficient to take only $RO_q$ instead of O into consideration for a specific query q. If a group G of objects can satisfy the coverage condition of definition 4, we say that G is a \textit{\textbf{feasible solution}} of query q. Put differently, KaGPC queries return the feasible solution with minimal costs.

\noindent
\begin{theorem}
\textbf{The KaGPC queries is NP-hard.}
\end{theorem}
Proof: We can reduce the classic WSC Problem to the KaGPC problem. A typical instance of the WSC problem consists of an universe $U=\{1,2,3,...,n\}$ of n elements and a family of sets $S=\{S_1,S_2,S_3,...,S_m\}$, where $S_i \subseteq U$ and each $S_i$ is associated with a positive cost $C_{S_i}$. The decision problem is to decide if we can find a subset $F$ of $S$ such that $\underaccent{S_i \in F}{\cup}S_i=U$ and such that its cost $\sum \limits_{S_i \in F}C_{S_i}$ is minimized.

To reduce the WSC problem to KaGPC queries q. We map each element of $U$ corresponding to a query keyword, that each set $S_i$ corresponding to a spatial object $o_i$, and each positive cost $C_{S_i}$ as the $wd(o_i,q)$. For each keyword associated with $o_i$, we set the integer value to 1. Besides we hypothesize the preference vector $P=\{1,0,0,0,0\}$ and the threshold is 1. Clearly, there is a solution to weighted set cover problem if and only if there is solution to query q.



